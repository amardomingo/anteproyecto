\documentclass[a4paper,11pt]{report}
\usepackage[T1]{fontenc}
\usepackage[utf8]{inputenc}
\usepackage{lmodern}
\usepackage{graphicx}
\usepackage{geometry}
\usepackage{hyperref}
\usepackage{url}
\usepackage[spanish]{babel}

\newcommand{\HRule}{\rule{\textwidth}{0.5mm}}
\begin{document}
\begin{titlepage}
\newgeometry{top=3cm}
\begin{center}
\textsc{ \large Universidad Politécnica de Madrid}
\end{center}  
\begin{figure}[!htbp]
\centering
   \includegraphics[width=0.5\textwidth]{images/logoetsit.png}
\end{figure}
\begin{center}

\textsc{ \large Escuela Técnica Superior }\\[0cm]
\textsc{ \large de Ingenieros de Telecomunicación}\\[2cm]
  
\textsc{ \LARGE Anteproyecto fin de carrera}\\[1cm]

\HRule \\[0.4cm]
{ \LARGE \bfseries Desarrollo de un Asistente integrado}\\[0.2cm]
{ \LARGE \bfseries con un Sistema de Indexación}\\[0.2cm]
{ \LARGE \bfseries Semántica de Información}\\[0.4cm] 

\HRule \\[4cm]

\textbf{ \LARGE Octubre 2014}\\[6cm]

\end{center}

\noindent
\begin{minipage}{0.4\textwidth}
\begin{flushleft} \large
\emph{Alumno:}\\
Alberto \textsc{Mardomingo}
\end{flushleft}
\end{minipage}%
\begin{minipage}{0.6\textwidth}
\begin{flushright} \large
\emph{Tutor:} \\
Carlos A. \textsc{Iglesias}
\end{flushright}
\end{minipage}

\restoregeometry
\end{titlepage}

\chapter*{Introducción}

Cada vez un número mayor de aplicaciones proporcionan una forma de acceso a diferentes tipos de información, muchas veces publicada sin ningún tipo de coordinación entre los diferentes agentes. Para solventar este problema, surge la inciativa de Web Semántica, con la idea de dirigir la evolución de la Web, y permitir que los usuarios puedan encontrar, compartir, combinar y estudiar la información de una forma sencilla. A pesar de que el objetivo inicial -que cualquier persona pudiese publicar información especializada- no se ha cumplido completamente, en la actualidad gran cantidad de organizaciones publican grandes cantidades de información, a través de portales especialidados para ello.

Dada la gran cantidad de información disponible, es deseable un sistema que permita interactuar con ella, indexarla, y realizar busquedas. Para la interacc



Qué es la web semántica, indexación de la información y las formas de interactuar con los sistemas

\chapter*{Objetivos}
¿Qué vamos a hacer en el proyecto?

\chapter*{Métodos y Fases del proyecto}
\begin{enumerate}
  \item Estado del arte
  \item Fase de aprendizaje
  \item Diseño de la solución
  \item Implementación
  \item Pruebas
  \item Demo
\end{enumerate}

\chapter*{Medios}

java, eclipse, vim, servidores...

\nocite{*}
l\bibliographystyle{ieeetr} 
{
\small
\bibliography{bibtex/pfc_javi}
}
\end{document}
